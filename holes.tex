\usepackage{calc}
%%%%%%%%%%%%%%%%%%%%%%%%%%%%%%%%%%%%%%%%%%%%%%%%%%%%%%%%%%%
%%% holes: stolen from Literate Agda paper.
%Define a reference depth. 
%You can choose either relative or absolute.
%--------------------------
\newlength{\DepthReference}
\settodepth{\DepthReference}{g}%relative to a depth of a letter.
%\setlength{\DepthReference}{6pt}%absolute value.

%Define a reference Height. 
%You can choose either relative or absolute.
%--------------------------
\newlength{\HeightReference}
\settoheight{\HeightReference}{T}
%\setlength{\HeightReference}{6pt}

%--------------------------
\newlength{\Width}%

\newcommand{\MyColorBox}[2][red]%
{%
    \settowidth{\Width}{#2}%
    %\setlength{\fboxsep}{0pt}%
    \colorbox{#1}%
    {%      
        \raisebox{-\DepthReference}%
        {%
                \parbox[b][\HeightReference+\DepthReference][c]{\Width}{\centering#2}%
        }%
    }%
}
\colorlet{Hole}{Dandelion!25!white}

% the hole-in-term symbol for Chapter 3
% \newcommand{\hole}{\ensuremath{\quad\cdot\quad}}
\newcommand{\minihole}{\MyColorBox[Hole]{\ensuremath{\{\,\,\}_{?}}}{}}
\newcommand{\hole}{\ensuremath{\quad}\minihole\ensuremath{\quad}}
%%%%%%%%%%%%%%%%%%%%%%%%%%%%%%%%%%%%%%%%%%%%%%%%%%%%%%%%%%%


%%% Local Variables:
%%% mode: latex
%%% TeX-master: t
%%% End:
