\chapter{Compilateur}
\begin{preamble}
Nous avons présenté dans le chapitre précédent les étapes de 
compilation de manière succincte afin d'illustrer le résultat 
de la compilation globale d'une règle Maloya. Dans ce chapitre 
nous nous attachons à présenter les mécanismes de compilation 
détaillés et leur spécificités.
\end{preamble}
\chpsummary{Aperçu
}
{
Présentation de la sémantique des opérateurs EPL utilisés à l'issue de la compilation d'une règle en Maloya.;
Description de la sémantique des opérateurs Maloya et de leur schéma de compilation.
}

Notre langage repose sur des opérateurs couvrant le domaine des services sensibles au contexte. Ces opérateurs permettent de manipuler des concepts d'états et d'évènements et peuvent être composés afin de formuler des services couvrant les besoins des intervenants du domaine du maintien à domicile des personnes âgées.
Nous présentons le schéma de compilation pour chacun des opérateurs ainsi que leur sémantique formelle.

Puisque nos opérateurs sont traduits dans des formules EPL, nous devons expliquer tout d'abord la sémantique d'EPL.

\section{Sémantique informelle des opérateurs EPL}
Le langage EPL fournit des opérateurs permettant de décrire des motifs à identifier dans un flux d'évènements. Cette section décrit les clauses fournies par EPL et utilisées dans la compilation finale des règles Maloya. Il n'existe pas, à notre connaissance, une sémantique formelle des opérateurs EPL~; nous décrivons donc de manière informelle la sémantique opérationnelle de chaque opérateur d'après le manuel de référence, enrichi par nos propres tests lorsque celui-ci n'est pas suffisamment détaillé.
\newline

La clause {\bf pattern} définit l’ensemble des évènements à détecter, ainsi que leurs
contraintes logiques et temporelles.%  Pour cela, des opérateurs logiques de conjonction et
\newline

La clause {\bf where} spécifie l’ensemble de contraintes associées à la règle. Il s’agit essentiellement de filtres et de conditions s’appliquant aux attributs des évènements.
\newline

La clause {\bf timer:within}, utilisée à la suite de la clause ``where'', permet de terminer la sous-expression concernée. Cette clause teste si la sous-expression concernée devient vraie au bout d'une durée donnée. Dans le cas contraire, la sous expression sera considérée comme fausse. Cette clause nous permet de borner la durée d'un état par exemple.
\newline

L'expression {\bf timer:interval} permet d'attendre une période de temps donnée avant de considérer cette expression comme vraie. Cette construction nous permet d'évaluer la persistance d'un état dans le temps par exemple.
\newline

La clause {\bf every}, préfixant une expression, indique que l'expression doit redémarrer sitôt évaluée à vrai ou faux. En effet, avec le comportement par défaut d'un motif EPL, sans le mot-clé ``every'', sitôt que la sous expression est évaluée à vrai ou faux, elle s'arrête.
\newline

La clause {\bf followed-by}, noté ``{\em ->}'', spécifie que l'expression à gauche de l'opérateur doit d'abord être vraie, et seulement après, l'expression de droite est évaluée pour trouver les évènements correspondants~: {\tt X~->~Y}.
Si l'expression {\tt Y} ne comporte pas de clause {\bf every}, comme expliqué ci-dessus, EPL ne reconnaîtra que sa première occurrence. Autrement dit, il ne peut pas y voir d'autres occurrences de {\tt Y} entre les {\tt X} et {\tt Y} identifiés par cette règle.
\newline

La clause {\bf not} inverse la valeur attendue pour qu'une expression réussisse. Les motifs d'expressions préfixés avec ``$not$'' sont considérés comme provisoirement vrais au départ, et deviennent définitivement faux quand l'expression incluse devient vraie. Dans les constructions EPL, cet opérateur est généralement utilisé en conjonction avec l'opérateur ``$and$''. Lors de notre compilation, nous utilisons toujours la construction {\tt X~and~not~Y}.
Cette expression impose qu'il n'y ait pas d'évènements {\tt Y} avant de reconnaître l'évènement {\tt X}.
\newline

La clause {\bf window}, associée à {\em create} permet de définir des fenêtres.  Les fenêtres sont des constructions qui permettent de définir des évènements utilisateur, identifiables par un nom, et contenant un nombre quelconque d'occurrences.
\newline

La clause {\bf insert into} insère des occurrences d'un évènement dans une fenêtre.


\section{Sémantique et compilation des opérateurs Maloya}
Dans la Section~\ref{sec:dsl:operator}, nous avons introduit les opérateurs du langage Maloya et présenté de manière informelle leur comportement. Nous décrivons maintenant leur compilation en pseudo-code EPL, c'est-à-dire en séquence d'évènements, ainsi que leur sémantique formelle. La sémantique est présentée avant la règle de compilation de chaque opérateur et est formalisée sous forme d'une fonction booléenne du temps~\paulcite{chakravarthy1994composite}. Nous notons $\llangle X\rrangle$ la fonction booléenne donnant la sémantique d'une expression Maloya $X$.  Cette fonction booléenne renvoie la valeur vrai exactement dans les moments où l'expression $X$ doit réussir.

Pour un évènement élémentaire, la fonction $\llangle e\rrangle$ renvoie vrai pour tous les moments où l'évènement $e$ se produit. Pour un état, la fonction $\llangle s=v\rrangle$ renvoie vrai à tous les moments où la dernière valeur fournie par le capteur $s$ est $v$.
La notation utilisée pour notre formalisation ne traitant que des évènements, nous exprimons les états en fonction de leurs évènements de début et de fin. Nous notons ainsi $[$E$_{sb},$E$_{se}]$ les évènements de début et de fin de l'état S.% (possiblement indicé).

Enfin, pour définir le schéma de compilation des opérateurs, nous notons $\llbracket X\rrbracket$, le code EPL produit pour une expression $X$ du langage Maloya.

Notons que pour chaque évènement de base, c'est-à-dire les évènements d'interactions exprimés dans une règle Maloya, le code de cet évènement correspond à l'expression de cette interaction que nous notons $\llbracket e\rrbracket=e$. Les états étant décomposés en évènements à la compilation, ils ne sont pas concernés.

\subsection*{Precedes(e$_1$,e$_2$)}
{\em Precedes} décrit la séquence de deux évènements E$_1$ et E$_2$. 
$\operatorname{Precedes}(E_1,E_2)$ est reconnu lorsqu'une occurrence de E$_2$ succède immédiatement à une occurrence de E$_1$. Ceci implique qu'il n'existe pas d'autres occurrences de E$_1$ ou de E$_2$ entre ces deux instants. Cet opérateur est défini ainsi~:
\begin{small}
\begin{equation*}
\begin{split}
\llangle\operatorname{Precedes}(E_1,E_2)\rrangle(t)=(\exists t_1)(\forall t_2)(&E_2(t)\wedge \\
&(t_1<t)\wedge\\
&E_1(t_1)\wedge\\
&((t_1<t_2<t)\rightarrow \thicksim (E_1(t_2)\vee E_2(t_2))))
\end{split}
\end{equation*}
\end{small}

La séquence d'évènements résultant de la compilation de l'opérateur {\em Precedes} spécifie qu'un évènement {\tt e$_1$} doit être suivi par un évènement {\tt e$_2$}, en respectant la définition de l'opérateur Esper ``followed-by'', avec la condition supplémentaire qu'aucun autre évènement {\tt e$_1$} ne doit se produire entre temps. La clause {\bf every} permet de tester la règle pour chaque occurrence de {\tt e$_1$}. Notons que les fonctions auxiliaires de compilation telles que ``WindowIfComplex'' sont décrites dans la Section suivante.
\begin{figure}
\begin{lstlisting}[language=EPLPseudoCodeCompile]
$\llbracket$Precedes(e$_1$,e$_2$)$\rrbracket$=
  every WindowIfComplex(e$_1$) $\rightarrow$ 
    WindowIfComplex(e$_2$) 
      and not WindowIfComplex(e$_1$) 
\end{lstlisting}
\end{figure}

\subsection*{During(e,s)}
{\em During} filtre toutes les occurrences d'un évènement durant la persistance d'un état donné. Plus précisément, soient E, E$_{sb}$ et E$_{se}$ trois évènements, où $[$E$_{sb}$,E$_{se}]$ décrit les évènements de début et de fin d'un état S. L'opérateur signale toutes les occurrences de E qui se produisent pendant l'intervalle démarré par E$_{sb}$ et terminé par E$_{se}$. Plus précisément~:
\begin{small}
\begin{equation*}
\begin{split}
\llangle\operatorname{During}(E,[E_{sb},E_{se}])\rrangle(t)=(\exists t_1)(\forall t_2)(&E(t)\wedge   \\
&(t_1<t)\wedge\\
&E_{sb}(t_1)\wedge\\
&((t_1<t_2< t)\rightarrow \thicksim E_{se}(t_2)))
\end{split}
\end{equation*}
\end{small}

La compilation de l'opérateur {\em During} décrit une occurrence d'un évènement correspondant au début de l'état {\tt s} (\ie la mesure d'interaction qui donne la valeur {\tt v$_{sb}$}), suivie de toutes les occurrences de l'évènement {\tt e}, sans qu'il n'y ait eu d'occurrence de l'évènement correspondant à la fin de l'état {\tt s} (\ie la mesure d'interaction qui donne la valeur {\tt v$_{se}$}). La clause {\em every} permet de tester la règle pour chaque occurrence de l'état {\tt s}. La fonction ``BoundedWindowIfComplex'' garantit que si {\tt e} est issue d'une séquence d'évènements fenêtrée, cette séquence à commencé après le début de l'état {\tt s}.
\begin{figure}
\begin{lstlisting}[language=EPLPseudoCodeCompile]
$\llbracket$During(e,s)$\rrbracket$=
  every Becomes(s,1) $\rightarrow$ 
    every BoundedWindowIfComplex(e,Becomes(s,1)) 
    and not Becomes(s,0)
\end{lstlisting}
\end{figure}


\subsection*{Occurs(e,s)}
{\em Occurs} a une sémantique similaire à {\em During}, excepté qu'il reconnaît seulement la première occurrence d'un évènement durant la persistance d'un état donné.
Soient E, E$_{sb}$ et E$_{se}$ trois évènements, où $[$E$_{sb}$,E$_{se}]$ décrit les évènements de début et de fin d'un état S. L'opérateur signale un évènement au moment de la première occurrence de E, qui se produit pendant l'intervalle démarré par E$_{sb}$ et terminé par E$_{se}$. 
Plus précisément,
\begin{small}
\begin{equation*}
\begin{split}
\llangle\operatorname{Occurs}(E,[E_{sb},E_{se}])\rrangle(t)=(\exists t_1)(\forall t_2)(&E(t)\wedge \\
&(t_1<t) \wedge\\
&E_{sb}(t_1)\wedge \\
&((t_1<t_2<t)\rightarrow \thicksim (E_{se}(t_2) \vee E(t_2))))
\end{split}
\end{equation*}
\end{small}

La compilation de l'opérateur {\em Occurs} est semblable à celle de l'opérateur {\em During}, excepté que seule la première occurrence de l'évènement {\tt e} déclenche un évènement. 
\begin{figure}
\begin{lstlisting}[language=EPLPseudoCodeCompile]
$\llbracket$Occurs(e,s)$\rrbracket$= 
  every Becomes(s,1) $\rightarrow$ 
    BoundedWindowIfComplex(e,Becomes(s,1))
    and not Becomes(s,0)
\end{lstlisting}
\end{figure}


\subsection*{Occurs(s$_1$,s$_2$)}
Cette version de l'opérateur {\em Occurs} exprime la superposition partielle de deux états persistants. Il s'agit d'une disjonction de deux évènements {\em Occurs(e,s)}.
Soient $[$E$_{s_{1}b}$,E$_{s_{1}e}]$ les évènements de début et de fin d'un état S$_1$ et, $[$E$_{s_{2}b}$,E$_{s_{2}e}]$ les évènements de début et de fin d'un état S$_2$. 
\begin{small}
\begin{equation*}
\begin{split}
\llangle\operatorname{Occurs}([E_{s_{1}b},E_{s_{1}e}],[E_{s_{2}b},E_{s_{2}e}])\rrangle(t)=&\llangle Occurs(E_{s_{2}b},E_{s_{1}b},E_{s_{1}e})\rrangle(t)\vee\\
& \llangle Occurs(E_{s_{1}b},E_{s_{2}b},E_{s_{2}e})\rrangle(t)
\end{split}
\end{equation*}
\end{small}

La compilation de l'opérateur {\em Occurs} prenant deux états en argument définit une disjonction sur la version de {\em Occurs} qui reçoit un évènement et un état en argument.
Pour que l'opérateur déclenche un évènement, l'état {\tt s$_2$} doit commencer durant l'état {\tt s$_1$} ou l'état {\tt s$_1$} doit commencer durant l'état {\tt s$_2$}.
\begin{figure}
\begin{lstlisting}[language=EPLPseudoCodeCompile]
$\llbracket$Occurs(s$_1$,s$_2$)$\rrbracket$=
  $\llbracket$Occurs(Becomes(s$_2$,1),s$_1$)$\rrbracket$
  or 
  $\llbracket$Occurs(Becomes(s$_1$,1),s$_2$)$\rrbracket$
\end{lstlisting}
\end{figure}

\subsection*{Overlapping(s$_1$,s$_2$)}
{\em Overlapping} décrit le chevauchement de deux états S$_1$,S$_2$, où S$_1$ commence avant S$_2$. 
Soient $[$E$_{s_{1}b}$,E$_{s_{1}e}]$ les évènements de début et de fin d'un 
état S$_1$ et,  $[$E$_{s_{2}b}$,E$_{s_{2}e}]$ les évènements de début et de 
fin d'un état S$_2$. 
Cet opérateur signale un évènement lorsqu'une occurrence de E$_{s_{1}b}$ est suivie d'une occurrence de E$_{s_{2}b}$, elle-même suivie d'une occurrence de E$_{s_{1}e}$.
De plus, il n'existe pas d'instant entre E$_{s_{1}b}$ et E$_{s_{2}b}$ pendant lequel se produit E$_{s_{1}e}$. Enfin, il n'existe pas d'instant entre E$_{s_{2}b}$ et E$_{s_{1}e}$ pendant lequel se produit E$_{s_{2}e}$. Plus spécifiquement,
\begin{small}
\begin{equation*}
\begin{split}
\llangle\operatorname{Overlapping}([E_{s_{1}b},E_{s_{1}e}],[E_{s_{2}b},E_{s_{2}e}])\rrangle(t)=&\\
(\exists t_1)(\exists t_2)(\forall t_3)(& E_{s_{1}e}(t)\wedge \\
&(t_1<t_2<t)\wedge \\
&E_{s_{1}b}(t_1)\wedge\\
&E_{s_{2}b}(t_2)\wedge\\
&((t_1<t_3 < t)\rightarrow  \thicksim E_{s_{1}e}(t_3)) \wedge \\
&((t_2<t_3 < t)\rightarrow  \thicksim E_{s_{2}e}(t_3)))%\\
\end{split}
\end{equation*}
\end{small}

La séquence d'évènements de l'opérateur {\em Overlapping} reconnaît une occurrence de l'évènement de début de {\tt s$_1$}, suivie d'une occurrence de l'évènement de début de {\tt s$_2$}, sans qu'il n'y ait eu d'occurrence de fin de {\tt s$_1$}, suivie d'une occurrence de fin de {\tt s$_1$}, sans qu'il n'y ait eu d'occurrence de fin de {\tt s$_2$}.
\begin{figure}
\begin{lstlisting}[language=EPLPseudoCodeCompile]
$\llbracket$Overlapping(s$_1$,s$_2$)$\rrbracket$ = 
  every Becomes(s$_1$,1) $\rightarrow$ 
    Becomes(s$_2$,1) 
    and not Becomes(s$_1$,0) $\rightarrow$ 
      Becomes(s$_1$,0) 
      and not Becomes(s$_2$,0) 
\end{lstlisting}
\end{figure}

\subsection*{And(e$_0$,$\dots$,e$_n$)}
{\em And} exprime la conjonction d'un ensemble d'évènements {E$_1$,$\dots$,E$_n$}. 
L'opérateur signale un évènement la première fois que tous les évènements E$_i$ se sont produits. 
Cela implique que le denier évènement manquant, E$_{i_0}$, vient de se produire pour la première fois.
Cet opérateur est défini comme suit.
\begin{small}
\begin{equation*}
\begin{split}
\llangle\operatorname{And}(E_1, \dots , E_n)\rrangle(t)=(\exists t_{i=1;n})(\exists i_0)(\forall t')(& (t_1\leq t)\wedge \dots \wedge (t_n\leq t) \wedge \\
& E_1(t_1)\wedge \dots \wedge E_n(t_n)\wedge \\
& (t_{i_0}=t) \wedge \\
&((t'<t)\rightarrow \thicksim E_{i_0}(t')))
\end{split}
\end{equation*}
\end{small}
Pour que l'opérateur {\tt And} déclenche un évènement, il faut qu'une occurrence de chaque évènement qu'il prend en argument se soit produit.
\begin{figure}
\begin{lstlisting}[language=EPLPseudoCodeCompile]
$\llbracket$And(e$_1$, $\dots$ ,e$_n$)$\rrbracket$=
  WindowIfComplex(e$_1$) and $\dots$ and WindowIfComplex(e$_n$)
\end{lstlisting}
\end{figure}

\subsection*{Or(e$_0$,$\dots$,e$_n$)}
{\em Or} exprime la disjonction d'un ensemble d'évènements {E$_1$,$\dots$,E$_n$}. L'opérateur signale un évènement quand un évènement de cet ensemble se produit. 
La définition de cet opérateur est la suivante~:
\begin{small}
\begin{equation*}
\begin{split}
\llangle\operatorname{Or}(E_1, \dots , E_n)\rrangle(t)=E_1(t)\vee \dots \vee E_n(t)
\end{split}
\end{equation*}
\end{small}

{\em Or} définit une disjonction normale et déclenche à chaque occurrence d'un évènement {\tt e$_i$}. 
\begin{figure}
\begin{lstlisting}[language=EPLPseudoCodeCompile]
$\llbracket$Or(e$_1$, $\dots$ ,e$_n$)$\rrbracket$=
  EveryIfLeaf(e$_1$) or $\dots$ or EveryIfLeaf(e$_n$)
\end{lstlisting}
\end{figure}

\section{Fonctions internes au compilateur}
La compilation des opérateurs vers du pseudo-code EPL utilise certaines fonctions auxiliaires de compilation. Par exemple il peut être nécessaire de vérifier si une fenêtre doit être créée pour une opérande, en fonction de son type~; ou encore d'extraire l'évènement de début ou de fin d'un état.  Ces opérations conditionnelles sont factorisées dans les fonctions suivantes, internes au compilateur.

Notons que toutes ses fonctions acceptent en entrée une expression en représentation interne de Maloya et offrent en sortie une expression en pseudo-code EPL.
\subsection*{Becomes}%: $\mathds{S}\times\mathds{B}\rightarrow \mathds{E}$}
Cette fonction traduit un état par l'évènement marquant son début ou sa fin, selon le deuxième argument booléen.
\begin{figure}[!h]
\begin{lstlisting}[frame=bt]
  $Becomes(p=v,1)\rightarrow p=>v$
  $Becomes(p=v,0)\rightarrow p=>v', v' \neq v$
\end{lstlisting}
\caption{Fonction Becomes. Traduit un état en un évènement.}
\label{listing:becomes}
\end{figure}
\subsection*{WindowIfComplex}
Cette fonction décide si une fenêtre doit être créée, en vérifiant si le fils du n{\oe}ud courant est une séquence d'évènements.
La fonction retourne l'identifiant de la fenêtre, si elle est crée~; sinon, le code du fils lui-même.
\begin{figure}[!h]
\begin{lstlisting}[frame=bt]
  $WindowIfComplex(e)\rightarrow$
    if e is leaf
    then
      $\llbracket$e$\rrbracket$
    else
      createWindow(e)
\end{lstlisting}
\caption{Fonction WindowIfComplex. Cette fonction vérifie si une fenêtre doit être créée.}
\label{listing:windoifcomplex}
\end{figure}

\subsection*{BoundedWindowIfComplex}
Cette fonction vérifie si le fils doit être fenêtré, de la même manière que la fonction WindowIfComplex. Si une fenêtre est créée, un argument est ajouté à l'identifiant de la fenêtre afin d'assurer que l'évènement produit par celle-ci est bornée par le n{\oe}ud passé
en deuxième argument.
\begin{figure}[!h]
\begin{lstlisting}[frame=bt]
  $BoundedWindowIfComplex(e_1,e_2)\rightarrow $
    if e$_1$ is leaf
    then
      $\llbracket$e$_1\rrbracket$
    else
      let window_id=createWindow(e$_1$)
      window_id `(timestamp >' e$_2$ `.timestamp)'
\end{lstlisting}
\caption{Fonction BoundedWindowIfComplex. Cette fonction vérifie si une fenêtre dont les évènements seront bornés doit être créée.}
\label{listing:createwindow}
\end{figure}

\subsection*{CreateWindow}
Cette fonction 
calcule le code de l'évènement complexe {\tt e} et le stocke dans un attribut de {\tt e}. %l'encapsule dans une fenêtre EPL.
Elle retourne ensuite l'identifiant de la fenêtre nouvellement créée.
\begin{figure}[!h]
\begin{lstlisting}[frame=bt]
  $CreateWindow(e)\rightarrow$ 
    code$_e\leftarrow \llbracket$e$\rrbracket$;
    window_id$_e$
\end{lstlisting}
\caption{Fonction CreateWindow. Cette fonction calcule le code correspondant à l'évènement complexe qu'il a en argument et retourne l'identifiant de la fenêtre.}
\label{listing:createwindow}
\end{figure}

\subsection*{EveryIfLeaf}
Cette fonction vérifie si le fils est une feuille et dans ce cas, retourne son code précédé de la clause {\em every}~; sinon, elle retourne simplement le code du fils. 
\begin{figure}[!h]
\begin{lstlisting}[frame=bt]
  $EveryIfLeaf(e)\rightarrow $
    if e is leaf
    then 
      `every' $\llbracket$e$\rrbracket$
    else 
      $\llbracket$e$\rrbracket$
\end{lstlisting}
\caption{Fonction EveryIfLeaf. Cette fonction vérifie la clause {\em every} doit être ajoutée devant le code de l'évènement.}
\label{listing:windoifcomplex}
\end{figure}


\section{Compilation finale vers EPL}
L'étape finale de compilation du pseudo-code EPL vers la représentation finale EPL peut survenir en deux circonstances~: (1) une fois sur l'expression complète à compiler~;
(2) pour chaque fenêtre créée par la fonction ``CreateWindow'', expliquée précédemment. 

Ainsi, la construction 
%\newline
{\em ``select~from~pattern~[$\llbracket$e$\rrbracket$]''} est générée pour l'expression complète.
De la même manière, pour chaque fenêtre ``window\_id$_e$'', cette étape génère le code EPL implantant la fenêtre en utilisant les constructions EPL: %\newline
%\begin{itemize}
%\item

{\em ``create~window~window\_id$_e$''} 

%\item
{\em ``insert~into~window\_id$_e$''} 

%\item
{\em ``select~from~pattern~[code$_e$]''}

%\end{itemize} 
\noindent
De cette façon, le ``pattern'' contient le code calculé par l'étape de compilation précédente.

Cette étape de compilation instancie également les évènements dans le pseudo-code EPL selon la forme {\em StreamEvent} en utilisant la fonction TranslateEvent~\ref{listing:translateevent}.
De plus, cette fonction lie tous les évènements dans la formule EPL à un même domicile, ou spécifie un domicile. Ce lien avec un domicile peut être fait de deux manières~: (1) si un domicile particulier est spécifié dans la règle, alors l'identification du domicile est transmise à l'attribut ``user'' de chacun des évènements (\eg ``user=userId'')~; (2) si aucun domicile n'est spécifié, alors, pour lier toute la séquence à un même domicile, chaque évènement de la séquence prend l'attribut ``user'' du premier évènement de la séquence (\eg ``user=X.user'', X correspondant à l'identifiant du premier évènement de la séquence).

\subsection*{TranslateEvent}
Cette fonction traduit les évènements pseudo-code EPL en évènements structurés EPL à partir
des informations présentes dans la tables statique~\ref{listing:table_static_generique}.
\begin{figure}[!h]
\begin{lstlisting}[frame=bt]
  $TranslateEvent(p=>v,e)\rightarrow$ 
    p=StreamEvent(location=p.location,
                  kind=p.kind,
                  value=`v',
                  user=e`.user')
  $TranslateEvent(p=>v,userId)\rightarrow$ 
    p=StreamEvent(location=p.location,
                  kind=p.kind,
                  value=`v',
                  user=userId)
\end{lstlisting}
\caption{Fonction TranslateEvent. Cette fonction transforme les évènements en pseudo-code EPL vers des évènements structurés EPL.}
\label{listing:translateevent}
\end{figure}

\section{Synthèse}
La compilation d'une règle en langage Maloya s'effectue en deux étapes: depuis la représentation interne vers le pseudo-code EPL, et depuis le pseudo-code vers le code EPL final. 
La première étape constitue l'étape principale de la compilation. Elle consiste à appliquer le schéma de compilation de chacun des opérateurs suivant leur sémantique afin de compiler les concepts d'états, d'évènements, et les contraintes temporelles en une séquence d'évènements. 
La compilation vers le code final EPL permet principalement d'instancier les évènements de cette séquence en respectant la forme canonique {\em StreamEvent} et de lier chacune de ces informations de contexte à un même domicile.

On constate que tous nos opérateurs peuvent, en principe, être programmés directement dans le langage EPL. Cependant, on constate également que les formules dans ce langage peuvent s'avérer complexes. Ainsi, la ré-implantation récurrente des motifs représentés par des opérateurs comme {\em Occurs} ou {\em Overlapping} peut être une source importante d'erreurs. Au contraire, la compilation automatique de nos opérateurs depuis le langage Maloya rend ces motifs prédictibles dans leur fonctionnement et simplifie la programmation.

