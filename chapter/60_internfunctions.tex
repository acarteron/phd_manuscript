\section{Fonctions internes au compilateur}
La compilation des opérateurs vers du pseudo-code EPL utilise certaines fonctions auxiliaires de compilation. Par exemple il peut être nécessaire de vérifier si une fenêtre doit être créée pour une opérande, en fonction de son type~; ou encore d'extraire l'évènement de début ou de fin d'un état.  Ces opérations conditionnelles sont factorisées dans les fonctions suivantes, internes au compilateur.

Notons que toutes ses fonctions acceptent en entrée une expression en représentation interne de Maloya et offrent en sortie une expression en pseudo-code EPL.
\subsection*{Becomes}%: $\mathds{S}\times\mathds{B}\rightarrow \mathds{E}$}
Cette fonction traduit un état par l'évènement marquant son début ou sa fin, selon le deuxième argument booléen.
\begin{figure}[!h]
\begin{lstlisting}[frame=bt]
  $Becomes(p=v,1)\rightarrow p=>v$
  $Becomes(p=v,0)\rightarrow p=>v', v' \neq v$
\end{lstlisting}
\caption{Fonction Becomes. Traduit un état en un évènement.}
\label{listing:becomes}
\end{figure}
\subsection*{WindowIfComplex}
Cette fonction décide si une fenêtre doit être créée, en vérifiant si le fils du n{\oe}ud courant est une séquence d'évènements.
La fonction retourne l'identifiant de la fenêtre, si elle est crée~; sinon, le code du fils lui-même.
\begin{figure}[!h]
\begin{lstlisting}[frame=bt]
  $WindowIfComplex(e)\rightarrow$
    if e is leaf
    then
      $\llbracket$e$\rrbracket$
    else
      createWindow(e)
\end{lstlisting}
\caption{Fonction WindowIfComplex. Cette fonction vérifie si une fenêtre doit être créée.}
\label{listing:windoifcomplex}
\end{figure}

\subsection*{BoundedWindowIfComplex}
Cette fonction vérifie si le fils doit être fenêtré, de la même manière que la fonction WindowIfComplex. Si une fenêtre est créée, un argument est ajouté à l'identifiant de la fenêtre afin d'assurer que l'évènement produit par celle-ci est bornée par le n{\oe}ud passé
en deuxième argument.
\begin{figure}[!h]
\begin{lstlisting}[frame=bt]
  $BoundedWindowIfComplex(e_1,e_2)\rightarrow $
    if e$_1$ is leaf
    then
      $\llbracket$e$_1\rrbracket$
    else
      let window_id=createWindow(e$_1$)
      window_id `(timestamp >' e$_2$ `.timestamp)'
\end{lstlisting}
\caption{Fonction BoundedWindowIfComplex. Cette fonction vérifie si une fenêtre dont les évènements seront bornés doit être créée.}
\label{listing:createwindow}
\end{figure}

\subsection*{CreateWindow}
Cette fonction 
calcule le code de l'évènement complexe {\tt e} et le stocke dans un attribut de {\tt e}. %l'encapsule dans une fenêtre EPL.
Elle retourne ensuite l'identifiant de la fenêtre nouvellement créée.
\begin{figure}[!h]
\begin{lstlisting}[frame=bt]
  $CreateWindow(e)\rightarrow$ 
    code$_e\leftarrow \llbracket$e$\rrbracket$;
    window_id$_e$
\end{lstlisting}
\caption{Fonction CreateWindow. Cette fonction calcule le code correspondant à l'évènement complexe qu'il a en argument et retourne l'identifiant de la fenêtre.}
\label{listing:createwindow}
\end{figure}

\subsection*{EveryIfLeaf}
Cette fonction vérifie si le fils est une feuille et dans ce cas, retourne son code précédé de la clause {\em every}~; sinon, elle retourne simplement le code du fils. 
\begin{figure}[!h]
\begin{lstlisting}[frame=bt]
  $EveryIfLeaf(e)\rightarrow $
    if e is leaf
    then 
      `every' $\llbracket$e$\rrbracket$
    else 
      $\llbracket$e$\rrbracket$
\end{lstlisting}
\caption{Fonction EveryIfLeaf. Cette fonction vérifie la clause {\em every} doit être ajoutée devant le code de l'évènement.}
\label{listing:windoifcomplex}
\end{figure}
